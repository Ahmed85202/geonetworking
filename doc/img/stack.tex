\documentclass[crop,tikz,convert={outext=.svg,command=\unexpanded{pdf2svg \infile\space\outfile}},multi=false]{standalone}[2012/04/13]

%\documentclass[tikz,border=10pt]{standalone}
                              % Reproservice prints on G4 which is about 80% of A4.
                              % With this scale, 12pt font becomes 9.6pt.
                              % https://www.chalmers.se/insidan/SV/arbetsredskap/kommunikation/tryck-och-layout/mallar-for-trycksaker/avhandlingens-utformning

\synctex=1                    % instead of "pdflatex -synctex=1 main"

\usepackage[utf8]{inputenc}   % unicode letters in the input tex document

\usepackage[T1]{fontenc}      % unicode letters in the output pdf

%\usepackage[pdftex]{graphicx} % images

\usepackage{color}

%\usepackage[usenames,dvipsnames]{xcolor}

\usepackage{tikz}             % figures drawn with LaTeX code

\usepackage{pgfplots}         % plots in LaTex, without Gnuplot
\pgfplotsset{compat=1.10}    % set compatibility to version 1.5.1 (default is 1.2)
\usepgfplotslibrary{groupplots}


% % % Tikz libraries and definitions
\usetikzlibrary{intersections}
\usetikzlibrary{decorations.markings} % accepting states by double circles
\usetikzlibrary{arrows}
\usetikzlibrary{automata}
\usetikzlibrary{shapes.misc}          % rectangle split shape
\usetikzlibrary{patterns}             % hatching pattern
\usetikzlibrary{plotmarks}            % marks on the plots
\usetikzlibrary{calc}                 % coordinate calculations ($ n1 + (0:1cm) $)
\usetikzlibrary{shapes.multipart}

\begin{document}

\tikzstyle{block} = [rectangle, rounded corners, minimum width=3cm, minimum height=1cm,text centered, thick, draw=black]

\tikzstyle{arrow} = [->,>=stealth',font=\scriptsize,rounded corners]

\newcommand{\shift}{0.28cm}
\newcommand{\bigshift}{2cm}

\newcommand{\itsstackwidth}{1cm}
%\begin{figure}
  \centering
  \begin{tikzpicture}[node distance=2cm, block/.append style={minimum width=4cm}]
  

  % Facilities
  \node [block, node distance=3cm, text width=2.6cm] (udp) {Vehicle adapter / Uppertester};

  \node [block, below of=udp, node distance=1.1cm] (cam) {CAM-DENM};

  \node [block, below of=cam, node distance=1.1cm] (asn1) {ASN.1 UPER};

  % Facilities box
  \draw ([xshift=-\shift, yshift=\shift]udp.north west) [block] rectangle  ([xshift=\shift,yshift=-\shift]asn1.south east);  

  % Facilities label
  \node [anchor=south west,rotate=90] (access) at ([xshift=-\shift, yshift=-\shift]asn1.south west) {Facilities};


  % Network
  \node [block, below of=asn1, node distance=1.7cm] (btp) {BTP};

  \node [block, below of=btp, node distance=1.1cm] (gn) {GeoNetworking};

  % Network box
  \draw ([xshift=-\shift, yshift=\shift]btp.north west) [block] rectangle  ([xshift=\shift,yshift=-\shift]gn.south east);  

  % Network label
  \node [anchor=south west,rotate=90] (access) at ([xshift=-\shift, yshift=-\shift]gn.south west) {Net\,\&\,Transp.};

  %%%%%%%%
  % Access
  %%%%%%%%
  % UDP-to-Eth
  \node [block, below of=gn, node distance=2.3cm] (udp2eth) {UDP-to-Eth};

  % Wlan
  \node [block, below of=udp2eth, node distance=2cm, minimum width=1.5cm, xshift=-4*\shift] (wlan) {\texttt{wlan}};

  % Tap
  \node [block, below of=udp2eth, node distance=2cm, minimum width=2cm, xshift=3.5*\shift] (tap) {\texttt{tap}};

  % OpenVPN client
  \node [block, below of=tap, node distance=1.1cm, minimum width=2cm, text width=1.8cm] (openvpn-client) {OpenVPN \\ Client};

  % Access box
  \draw ([xshift=-\shift, yshift=\shift]udp2eth.north west) [block] rectangle  ([xshift=\shift,yshift=-\shift]openvpn-client.south east -| udp2eth.north east);  

  % Access label
  \node [anchor=south west,rotate=90] (access) at ([xshift=-\shift, yshift=-\shift]openvpn-client.south east -| udp2eth.north west) {Access};

  % Antenna from wlan
  \draw (wlan.west) -| ++(-1,0.5) node [inner sep=0pt,outer sep=0pt] (antenna) {} -- ++(90:0.5cm);
  \draw (antenna) -- ++(60:0.6cm);
  \draw (antenna) -- ++(120:0.6cm);

  % GN - UDP-to-Eth
  \draw [arrow, <->] (gn.south) node [name=gnLeft] {} -- node [anchor=west] {UDP} (gnLeft |- udp2eth.north);
  % \draw [arrow] ([xshift=-0.5*\bigshift]gn.south) node [name=gnLeft] {} -- (gnLeft |- udp2eth.north);
  % \draw [arrow, <-] ([xshift=0.5*\bigshift]gn.south) node [name=gnRight] {} -- (gnRight |- udp2eth.north);
  %\path (gn) -- node  {\textsc{UDP}} (udp2eth);


  % UDP-to-Eth - TAP
  \draw [arrow, <->] (udp2eth.south -| wlan.north) node [name=u2eLeft] {} -- (u2eLeft |- wlan.north);

  \draw [arrow, <->] (udp2eth.south -| tap.north) node [name=u2eRight] {} -- (u2eRight |- tap.north);

  \path (udp2eth) -- node [xshift=-0.5cm, text width=2.0cm,  text centered]  {\small \textsc{raw socket or pcap}} (tap);



  \end{tikzpicture}
%  \caption{ITS stack for both physical vehicle and for Interactive Test Tool. When stack is used for physical vehicle, \texttt{wlan} interface towards physical network card and antenna is used. When stack is used for Interactive Test Tool, \texttt{tap} interface is used towards OpenVPN client.\label{fig:its-stack-impl}}
%\end{figure}




\end{document}
